\documentclass[twocolumn]{article}
\usepackage{default}

\addbibresource{references.bib}
\graphicspath{{./figures/}}

\title{Handling Phase Defects and Non-Conductive Tissue}
\author[1]{Bjorn Verstraeten}
\affil[1]{Ghent University}
\date{\today}

\begin{document}

\twocolumn[
  \begin{@twocolumnfalse}
    \maketitle

    \begin{center}
      \resizebox{0.8\textwidth}{!}{\includegraphics{comparison_snapshot_overview.png}}
      \captionof{figure}{
        % Snapshot of a simulation with the Fenton-Karma model that
        % contains discontinuities and holes. An analysis is done with a naive
        % implementation of phasemapping (left) and with the proposed approach
        % (right). The values of the mesh correspond the phase field, of which a
        % value of \(\pi\) corresponds with the excitation wavefront.
        % Discontinuities, boundaries and rotors are annotated with red if a
        % counterclockwise rotation is found and in blue for clockwise rotation.
        % Discontinuities and boundaries that do not show any rotation are
        % annotated in white.
      }
      \label{fig:comparison_snapshot_overview}
    \end{center}

    \begin{abstract}
      \noindent Rotational activity is one of the mechanisms behind
      cardiac arrhythmias,
      and it is therefore important to accurately localize such activity in
      cardiac tissue. A standard method for this, is called phasemapping.
      While widely used for analysing rotors in cardiac simulations,
      phasemapping struggles to correctly analyse activation maps with large
      non-conductive regions and/or conductive blocks. To illustrate these, we
      examine a simulation that exhibits these treats, and demonstrate how
      each of them can be addressed.
    \end{abstract}

  \end{@twocolumnfalse}
  \vspace{2em}
]

\section{Introduction}\label{introduction}

We present a simple simulation in figure~\ref{fig:comparison_snapshot_overview}.
By visually inspecting the simulation, a linear rotor appears to turn
counterclockwise in the bottom left \autocite{arno2021a}.
Looking at the bottom right,
you can see two wavefronts soon to collide with each other.
At the top we find a more complex pattern.
There is a non-conductive region around which three wavefronts are moving.
One might recognize this pattern as a critical boundary with
near-complete rotation;
a pattern that plays an important role during ablation therapy.
\autocite{duytschaever2024atrial, santucci2024identification}.

While for this simple simulation, a visual inspection is reasonable,
it is not considered as an optimal solution as it is subject to human mistakes.
Instead, it would much more preferable to have a tool
that would act as some kind of spell check where the error,
in this case the rotational activity, is highlighted.

A potential candidate for such tool is called phasemapping,
which is a method that is often used for rotor detection (CITATION).
While a naive implementation of phasemapping would detect rotors
reasonably well,
it struggles to analyse simulations with non-conductive regions or
conductive blocks,
such as the simulation picture above.
This is not because the idea behind phasemapping is lacking.
Rather, it's because a naive approach mistreats such simulations.

\subsection{What is the phase index?}\label{what-is-the-phase-index}

Phasemapping relies on a metric called the phase index,
which is formally defined by

\begin{equation}
  I = \frac{1}{2\pi}\oint_C \nabla \phi \cdot d\bm{l} ,
  \label{eq:phase-index}
\end{equation}

\noindent where $C$ is a closed curve
and $\phi$ is referred to as the phase field.
The phase field $\phi$ can be any scalar field of which the values are periodic.
The phase index can be used to analyse cardiac activation maps
by realizing that the state of a cardiac cell changes periodically;
a cell can go from resting state to an excited state,
and after a while it repolarizes back to its resting state (CITATION).
From now on, the phase $\phi$ will denote a value between $0$ and $2\pi$
that represents the state of a cell.

To get an understanding of what the phase index represents,
we can imagine moving across the curve
and looking at how the phase changes.
Once we get back to our starting point,
two possibilities can occur:
Either the phase of the starting point has the same value as at the beginning,
or the values has been shifted be a multiple of $2\pi$.
Note that all possibilities represent the same state.
However, in the last case, the phase index will be non-zero.

When analysing an electro-anatomical map, we often want to know if there
is some rotational activity present, since this is one of the mechanisms
behind cardiac arrhythmia. To express rotational activity a bit more
quantitative, we can state that, if there is a closed curve \(C\) around
which the electrical excitation travels, there is rotational activity
across that curve.

We can get an even clearer expression if we describe the state of each
point at a certain time with a single variable. Considering that the
state of each point changes periodically, the variable can be any scalar
that is lying onto the unit circle. This variable is referred to as the
phase \(\Phi\).

For a phase field \(\Phi(x, t)\), which describes the state of each
point in a two-dimensional orientable manifold M (e.g.~a plane or the
surface of a sphere), it holds that if \(\Phi\) is smooth across \(C\),
and we follow \(\Phi\) along \(C\) until we end up back at the starting
point, The value will now be the same, or it will be shifted by
\(2\pi\). This expression can also be written down in a mathematical
formula.

\noindent where \(I\) is called the phase index of the curve. Using Stokes'
theorem, the phase index can also be written as

\begin{equation}
  I = \frac{1}{2\pi} \iint_S \nabla \times (\nabla \Phi) \cdot d\bm{S}
  \label{eq:phase-index-stokes}
\end{equation}

\noindent In case \(\Phi\) is differentiable across S, the index \(I\) is zero.
Which means that \(I\) is only non-zero if a singular point is present
in S. \autocite{herlin2012reconstruction} This singular point is exactly
what a rotor tip or spiral core is being identified with.

\subsection{Properties of the phase
index}\label{properties-of-the-phase-index}

The first thing we want you to notice, is that the phase index is a
property of a closed curve. This can be any closed curve, but for our
purposes, we will stick to closed curves that do not intersect
themselves.

Secondly, since the phase index is an integer and the manifold is
differentiable, the curve can be continuously deformed without altering
the phase index. Therefore, the phase index is conserved for continuous
deformations, which is why it is called a topological charge. This
property allows us to identify a phase index to a singular point as
follows: The phase index of a singular point is the phase index of any
closed curve that surrounds that point and nothing else.

This idea can be generalized by stating that the phase index of a closed
curve is equal to the total phase index of everything in that curve. For
example, if we draw a curve around multiple singular points \(x_1\),
\ldots, \(x_N\), the phase index \(I_C\) of the curve is equal to that
of the sum of the phase indices \(I_n\) of all singular points inside.

\begin{equation}
  I_C = \sum^N_{n=1} I_n
  \label{eq:index-curve}
\end{equation}

\noindent To build further on this idea, suppose that we have defined a phase
field \(\Phi\) on a closed surface \(S\). If we now draw a curve on that
surface with nothing in it, its phase index will be zero. Since \(S\) is
a closed surface, this curve also encircles everything, and thus, we can
write

\begin{equation}
  \sum_{x_n \in S} I(x_n) = 0
  \label{eq:index-theorem}
\end{equation}

\noindent This is the same as stating that the total topological charge is
conserved on a closed surface.

In order to extend this conservation law for surfaces with boundaries or
holes we can imagine to fill up these holes to get a closed surface and
use their boundaries as the closed curve to calculate the phase index.
For a more thorough explanation we refer to
\autocite{herlin2012reconstruction, davidsen2004topological}. This
extension then gives us the following equation:

\begin{equation}
  \sum_{x_n \in S} I(x_n) = \sum_{H_m \in S} I(H_m)
  \label{eq:index-theorem-extended}
\end{equation}

\noindent where \(x_n\) are the singular points in \(S\) and \(H_m\) are the
holes. From now on we will refer to this as the index theorem.

\subsubsection{How to handle conduction blocks or phase
defects}\label{how-to-handle-conduction-blocks-or-phase-defects}

So far we have defined the phase index for singular points and holes,
but now we need to remind ourselves about which requirements need to be
fulfilled in order to do so. For our purposes, the most important one is
that the phase field needs to be smooth across the closed curve. One
example where this is not fulfilled are singular points. Another example
that applies to cardiac tissue is when the wavefront of the excitation
wave hits a refractory region. This forms a discontinuity or defect in
the form of a line. A full discussion about this phenomenon can be found
in \autocite{tomii2021spatial}, but for our purpose it is enough to define a
phase defect as a discontinuity in the phase field.

Figure~\ref{fig:index_calculation} show an example on how to adjust the curves
when phase defects and holes are present.

\begin{figure}[ht]
  \centering
  \includefig{index_calculation}
  \caption{Example of a bounded planar domain with holes. The domain
    contains singular points at \(\pmb{x}_1\), \ldots, \(\pmb{x}_n\) and
    holes \(H_1\), \ldots{} \(H_{m-1}\), with the outer boundary
    representing the final hole \(H_m\). Curves are drawn around each
    singular point and hole with the arrow represent the direction of
    integration. In addition, a phase defect is presented together with a
  curve over which to integrate to get the phase index of the defect.}
  \label{fig:index_calculation}
\end{figure}

\subsection{Detecting rotational activity in cardiac
tissue}\label{detecting-rotational-activity-in-cardiac-tissue}

A well-known method that exploits the idea of phase indices is called
phasemapping. With phasemapping, the cardiac activation map is scanned
across with a small area and the phase index is computed along the
circumference of this area. Since this method is mostly used while
analysing rotors, the scanning area is made as small as possible in
order to ensure that only one singular point lies within the area.

We argue that using phasemapping will give an incomplete analysis
because the cardiac can have phase defects and boundaries for which the
points that lie on them cannot be properly analysed using this method
Figure~\ref{fig:index_calculation} shows that you cannot draw a
closed curve solely around
such points that is smooth. In other words, there is no way of analysing
all points separately and the next best thing that we can do, is
calculating the phase index for the boundaries and the phase defect as a
whole.

\vspace{2em}

\noindent In this paper, we will summarize the idea behind phasemapping, putting
emphasis on the requirements for which the standard implementation gives
the correct result. We will then formally define conductive blocks and
showcase how to extend standard phasemapping in order to treat
conductive block and non-conductive regions properly.

\section{Methods}\label{methods}

There exists a numerous amount implementations and methods to detect
rotational activity \autocite{pikunov2023the, gurevich2019robust,
li2020standardizing}
that it is almost impossible to create
comparative study or full review. Therefore, we have decided to create a
case study. We will analysis single simulation that contains boundaries
and phase defects, and simple enough to visually confirm.

\subsection{Setup of the Simulation}\label{setup-of-the-simulation}

As it is not required to have realistic geometries to induce phase
defects, We chose to create a simulation on a homogeneous 2D grid. This
also has the benefit to get a full overview of the simulation at once
and easily place snapshots in the paper.
The simulation is created using Finitewave
(\url{https://github.com/finitewave/Finitewave}), an open-source Python
package for a wide range of tasks in modelling cardiac electrophysiology
using finite-difference methods. The main argument for using Finitewave
is its clear and transparent implementation of the models which allows
us to verify its correctness. Furthermore, the intuitive interface and
lots of examples make it easy to create, evaluate, and adjust
simulations.

The cell model used for the simulation is the Fenton-Karma model. Since
the research question of this paper does not involve the effect of ionic
channels on the dynamics, it is more fitting to use a phenomenological
model which is less computationally heavy and has simpler cell dynamics.
We choose the Fenton-Karma model specifically since it was the easiest
to create phase defects without tweaking the model's parameters.

The 2D mesh is first pre-paced with 10 planar stimuli and an interval of
200 time steps between them. Next, a boundary is added inside the mesh
and a second square stimulus is applied to induce rotational activity.
To ensure that the rotor fits on the mesh, but without increasing
computational time, we lowered the conductivity of the mesh. Finally, we
added a phase defect using a third stimulus in a small region just
before it goes out of refractory.

\subsection{Implementation of
phasemapping}\label{implementation-of-phasemapping}

To highlight the influence of phase defects and boundaries, we will
compare two implementations of phasemapping: one where the simulation is
scanned for singularities without addressing phase defects and ignoring
boundaries, which we will refer to as naive phasemapping, and another
were we take into account phase defects and boundaries. This second
implementation is basically an extension of the first one, so we will
refer to it as extended phasemapping.

\subsection{Naive Phasemapping}\label{naive-phasemapping}

The first step was to convert the action potential to a phase field.
This was done using taking the angle Hilbert transform of the action
potential as suggested by \autocite{bray2002considerations}. Additionally,
we made sure that the peak of the action potential corresponds to
\(\pi\) (see Figure~\ref{fig:action_potential_and_phase}),
so that it becomes straightforward to compare the phase field with
the action potential.

\begin{figure}[ht]
  \begin{center}
    \includegraphics[width=\columnwidth]{figures/action_potential_and_phase.png}
  \end{center}
  \caption{
    Time series of the action potential of one cell (orange)
    and the calculated phase (blue).
  }\label{fig:action_potential_and_phase}
\end{figure}

Next, we will create a triangulated mesh and compute the phase index for
each triangle at each time step. In theory, the cells of the mesh could
be any polygon, but triangulated meshes are quite common and choosing
so, made the code simpler.

For any polygon, the phase index can be calculated by counting the
number of phase jumps. An algorithm for this will look like:

\begin{enumerate}
    \def\labelenumi{\arabic{enumi}.}
    \tightlist
  \item
    Compute the phase difference \(\Delta\Phi\) of all edges.
  \item
    Count the number of times \(\Delta\Phi\) is bigger than \(\pi\)
    (positive phase jump).
  \item
    Count the number of times \(\Delta\Phi\) is smaller than \(-\pi\)
    (negative phase jump).
  \item
    Calculate the index with \(I = P - N\) with \(P\) and \(N\) the number
    of positive and negative phase jumps respectively.
\end{enumerate}

\subsection{Extended Phasemapping}\label{extended-phasemapping}

To extend the naive approach, the first thing to do is localizing the
phase defects. Remember that phase defects are defined as
discontinuities in the phase field. In a triangulated mesh this would
manifest as a big phase difference on the edges. Therefore, the most
straightforward thing to do is setting a threshold \(d\) so that an edge
with a phase difference \(\Delta\Phi\) that satisfies
\(d<\Delta\Phi<2\pi-d\) would be considered as a phase defect. We have
found that a value of \(d=0.08\pi\) works best for this simulation, but
keep in mind that this threshold depends highly on the parameters of the
simulation.

Once all phase defects are located, the cells that contain at least one
phase defect are removed. This will create holes in the mesh, allowing
us to treat phase defect and boundaries as the same.

Finally, the boundaries and holes are extracted as polygons. The naive
phasemapping approach (see previous section) is then applied to these
polygons together with the remaining cells.

\section{Results}\label{results}

Looking back at Figure~\ref{fig:comparison_snapshot_overview}, it is
clear that the naive
and extended approach does not give the result. Notice that the naive
approach localizes rotational activity in the bottom right and left
while the extended approach localizes rotational activity in the bottom
right and around the non-conductive region at the top. Moreover, the
analysis of the naive approach does not satisfy the index theorem given
in equation~\ref{eq:index-theorem-extended}. This should raise some
suspicion that this analysis
inaccurate.

\begin{figure}[ht]
  \centering
  \includegraphics[width=\columnwidth]{zoom_rotor.png}
  \caption{Zoom of the bottom left of the simulation, showcasing a
    counterclockwise rotation. Left: A snapshot at time step 172, with the
    colours represent the phase of the points at that time step. Rotors and
    critical cycles are annotated in red. Right: The phase density map taken
    across the entire time of the simulation. The colours represent the
    number of time steps that a point was annotated. A log scale was used to
    enhance visibility. The results of the naive approach are displayed at
  the top, and the results of the extended approach the bottom.}
  \label{fig:zoom-rotor}
\end{figure}

First, let's focus on the bottom left (see Figure~\ref{fig:zoom-rotor}).
Both approaches identify rotational activity in this region, and by
comparing both point density maps, it is clear that they also recognize
a similar drift. However, the extended approach detects a phase defect
at the centre of the rotational activity, indicating that this is a
linear rotor, while this information is not present in the analysis of
the naive approach.

\begin{figure}[ht]
  \centering
  \includegraphics[width=\columnwidth]{zoom_defect.png}
  \caption{Zoom of the bottom right of the simulation, showcasing a phase
    defect without rotation. From left to right, snapshots are taken at time
    steps 157, 177 and 235. The colours represent the phase of the points at
    that time step. (Counter)clockwise rotation is annotated in red (blue)
    and phase defect without rotation is annotated in white. The results of
    the naive approach are displayed at the top, and the results of the
  extended approach the bottom.}
  \label{fig:zoom-defect}
\end{figure}

Next, we shift focus to the bottom right (see
Figure~\ref{fig:zoom-defect}). Looking at the snapshots of the simulation,
the naive approach highlight two rotors in opposite direction, which
collide with each other before making a complete turn. In contrast, the
extended approach does not identify rotors in this region. Instead, it
finds a phase defect that is located between the two rotors. What
happened here is that the extended approach calculated the phase index
of both rotors together, and since these are of opposite sign, they
cancel each other out.

\begin{figure}[ht]
  \centering
  \includegraphics[width=\columnwidth]{zoom_reentry.png}
  \caption{Zoom of the top of the simulation, showcasing rotation around a
    non-conductive region. Only the results of the extended approach are
    shown, since the naive approach did not detect anything. From left to
    right and top to bottom, snapshots are taken at time steps 132, 172 and
    198. The colours represent the phase of the points at that time step and
  critical cycles with clockwise rotation are annotated in blue.}
  \label{fig:zoom-reentry}
\end{figure}

Finally, we look at the top of the simulation (see
Figure~\ref{fig:zoom-reentry}). We now have the opposite situation as
before at the bottom right: The extended approach finds rotational
activity around the non-conductive tissue while the naive approach does
not. A closer observation of the different snapshots reveals an
interesting pattern: The number of wavefronts around the non-conductive
tissue alternates between one and three. But what makes is even more
peculiar is that none of the wavefronts makes a complete turn. This
reminds us about the pattern seen in the bottom right where two rotors
collide with each other before completing a turn. Yet, this time the
total phase index is non-zero.

\section{Discussion}\label{discussion}

The presented simulation is nothing out of the ordinary. It has only
three main components: a linear core, rotation around a non-conductive
region and a phase defect. That is it. No anisotropic conductivity, no
fibres and no noise. The simulation is also of a good resolution, which
also favours phasemapping. Yet, it can cause trouble for phasemapping if
not treated properly.

As expected, phasemapping is good at finding rotors, which is why both
approaches find rotational activity at the bottom left. Nevertheless,
the rotor in this simulation has a linear core and, as discussed in
\autocite{arno2021a, tomii2021spatial}, is different from a rotor tip.
Discussing the phase defect instead of the rotor tip would also simplify
the description of the rotor movement. Notice that the rotor tip is
always attached to the phase defect. This means that you can split up
the movement of the rotor tip as a rotation around the phase defect and
the movement of the phase defect. What we are implying is that just the
movement of the phase defect is enough to describe the dynamics of the
system and that is generally less complex.

In contrast to the successful detection of rotors, we have demonstrated
that phase defects can cause false positives, as seen in
Figure~\ref{fig:zoom-defect}. Some people would dismiss these rotors tips
by stating that they do not make a full rotation, but this would require
a visual inspection or post-processing steps. Accounting for phase
defects dismisses these rotors already, reducing the necessary steps to
get to the correct interpretation.
However, phase defects do not tell the whole story. For example, looking
at the phase defect, the discontinuity disappears when the wavefront is
moving away from it, but the next wave still can't pass through it.
Since the phase space of cardiac tissue can exhibit interfaces or
defects, it is not necessarily guaranteed that phasemapping and waveback
-and front intersection points give the correct analysis. This could
explain the reason why using different thresholds to count phase jumps,
gives different results \autocite{li2020standardizing}.

Not much attention is given to non-conductive structures, but they do
play an import role in the clinical field. As for the pattern in the
presented simulation, it turns out that this used to be dismissed as not
important. However, recent studies have found that these play a critical
role in the success of ablation therapies
\autocite{duytschaever2024atrial, santucci2024identification, takigawa2019a}.
Our naive implementation did not detect anything around the
non-conductive tissue, but in case an implementation searches for rotors
by looking at the endpoints of the wavefront, it would detect either one
or three rotors. While a phase density map would then highlight the full
boundary, the interpretation would be a bit messy. Especially if you
consider rotors only to be true if they make a full rotation.
Calculating the phase index of the boundary gives a clearer
interpretation and is therefore preferred.

Hopefully, one notices that non-conductive structures, boundaries and
phase defects are all treated in the same manner.
In case such structures have a non-zero phase index,
it is quite natural to call them all as finite cores,
referring to the finite circumference that all of them possess.
We would like to highlight that no singular cores were found when
applying the extended approach.
This result aligns with the results from \autocite{tomii2021spatial, arno2021a},
that states that all rotors are found around a phase defect (CHECK
WHAT ARNO SAID).
This could imply that only finite cores are possible.
An implication that is also strengthened by the argument that the
tissue has a finite minimal wavelength,
or the smallest curve with a non-zero phase index has a finite circumference.

Finally, one could argue that the naive approach that was used,
was too simplistic and that current algorithms would do a better job at
analysing this simulation.
However, we have not encountered implementation of rotor detection
that explicitly account for this.
Rather than stating that the extended approach would yield better results,
we imply that the current state-of-the-art algorithms would benefit
from explicitly checking for phase defects.

\subsection{Future work}\label{future-work}

The main goal of our research is to develop analytical tools for physicians
that leverage theoretical insights.
We think that extending phasemapping to handle non-conductive tissue
and phase defects,
can be useful when examining activation maps of atrial tachycardia.
With that in mind, a few things need to be checked before
phasemapping can be used in the clinical field.

First up, the activation maps are generally speaking of lower resolution
and contain noise.
Previously, it was already shown that phasemapping does not perform well on
noisy and low-resolution data \autocite{lootens2024directed} (CITATION LOW-RES).
It would be interesting to replicated robustness analyses of
phasemapping methods using the extended approach.
Additionally, if one requires that only finite cores exist,
the robustness of phasemapping could be increased even further.

Secondly, we could apply fibrosis to the tissue.
We hypothesize that this would not form any problems,
since fibrotic tissue can be considered the same as non-conductive tissue.

Finally, we believe that there still exist some improvements in
constructing a phase field.
As shown in Figure~\ref{fig:action_potential_and_phase},
the phase that is created with the Hilbert transform is quite steep
around phase jumps,
which makes differentiating the phase jumps from the phase defects
more difficult.
Additionally, it does not account for that fact that cells can linger
in the resting state.
Some new definitions are proposed by~\cite{kabus2022numerical} that
could improve the construction of phases.

\section{Conclusion}\label{conclusion}

\begin{itemize}
    \tightlist
  \item
    Properly addressing phase defects and adapting the curve can make
    phasemapping suitable for analysing clinical cases.
  \item
    Refer to preregistration.
  \item
    Additionally, we encourage research groups to try out this simple
    example to see whether their detection method can analyse this map
    correctly. However, keep in mind Goodhart's Law
\end{itemize}

This paper is meant to serve as a theoretical backbone to understand how
phasemapping can be used in a clinical setting. Researcher who are
interested in testing these ideas, can have a look at our
preregistration (LINK!!!).

\section{Data availability}\label{data-availability}

\begin{itemize}
    \tightlist
  \item
    GitHub: code + source code
  \item
    Zenodo: figures, simulation data, video
  \item
    OSF: preregistration
\end{itemize}

Readers who would like to reproduce our results are referred to this
GitHub repository (LINK!!!), which contains the used code and some
further explanation.

\printbibliography
\end{document}

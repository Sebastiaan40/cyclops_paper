\documentclass[twocolumn]{article}
\usepackage{default}

\addbibresource{references.bib}
\graphicspath{{./figures/}}

\title{Handling Phase Defects and Non-Conductive Tissue}
\author[1]{Bjorn Verstraeten}
\affil[1]{Ghent University}
\date{\today}

\begin{document}

\twocolumn[
  \begin{@twocolumnfalse}
    \maketitle

    \begin{center}
      \resizebox{0.8\textwidth}{!}{\includegraphics{comparison_snapshot_overview.png}}
      \captionof{figure}{
        % Snapshot of a simulation with the Fenton-Karma model that
        % contains discontinuities and holes. An analysis is done with a naive
        % implementation of phasemapping (left) and with the proposed approach
        % (right). The values of the mesh correspond the phase field, of which a
        % value of \(\pi\) corresponds with the excitation wavefront.
        % Discontinuities, boundaries and rotors are annotated with red if a
        % counterclockwise rotation is found and in blue for clockwise rotation.
        % Discontinuities and boundaries that do not show any rotation are
        % annotated in white.
      }
      \label{fig:comparison_snapshot_overview}
    \end{center}

    \begin{abstract}
      \noindent Rotational activity is one of the mechanisms behind
      cardiac arrhythmias,
      and it is therefore important to accurately localize such activity in
      cardiac tissue. A standard method for this, is called phasemapping.
      While widely used for analysing rotors in cardiac simulations,
      phasemapping struggles to correctly analyse activation maps with large
      non-conductive regions and/or conductive blocks. To illustrate these, we
      examine a simulation that exhibits these treats, and demonstrate how
      each of them can be addressed.
    \end{abstract}

  \end{@twocolumnfalse}
  \vspace{2em}
]

\section{Introduction}\label{introduction}

% Doesn't flow, probably need more segue

We present a snapshot of a simple simulation in figure~\ref{fig:comparison_snapshot_overview}.
By visually inspecting the simulation, a linear rotor appears to turn
counterclockwise in the bottom left \autocite{arno2021a}.
Looking at the bottom right,
you can see two wavefronts soon to collide with each other.
At the top we find a more complex pattern:
there is a non-conductive region around which three wavefronts are moving.
One might recognize this pattern as a critical boundary with
near-complete rotation;
a pattern that plays an important role during ablation therapy.
\autocite{duytschaever2024atrial, santucci2024identification}.

While for this simple simulation, a visual inspection is reasonable,
it is not considered as an optimal solution as it is subject to human mistakes.
Instead, it would much more preferable to have a tool
that would act as some kind of spell check where the error,
in this case the rotational activity, is highlighted.
A potential candidate for such tool is called phasemapping,
which is a method that is often used for rotor detection (CITATION).
While a naive implementation of phasemapping would detect rotors
reasonably well,
it struggles to analyse simulations with non-conductive regions or
conductive blocks,
such as the simulation pictured above.
This is not because the idea behind phasemapping is lacking.
Rather, it's because a naive approach mistreats such simulations.

\subsection{What is the Phase Index?}\label{what-is-the-phase-index}

Phasemapping relies on a metric called the phase index,
which is formally defined by

\begin{equation}
  I = \frac{1}{2\pi}\oint_C \nabla \phi \cdot d\bm{l} ,
  \label{eq:phase-index}
\end{equation}

\noindent where $C$ is a closed curve
and $\phi$ is referred to as the phase field.
The phase field $\phi$ can be any scalar field of which the values are periodic.
The phase index can be used to analyse cardiac activation maps
by realizing that the state of a cardiac cell changes periodically;
a cell can go from resting state to an excited state,
and after a while it repolarizes back to its resting state (CITATION).
From now on, the phase $\phi$ will denote a value between $0$ and $2\pi$
that represents the state of a cell.

% Might need a picture to give more intuition

To get an understanding of what the phase index represents,
we can imagine moving across the curve
and looking at how the phase changes.
Once we get back to our starting point,
two possibilities can occur:
Either the phase of the starting point has the same value as at the beginning,
or the values has been shifted with a multiple of $2\pi$.
Note that all possibilities represent the same state.
However, in the last case, the phase index will be non-zero.
In other words, the phase index tells us how the state of a cell changes when traversing over a curve.
One possible curve for which the phase index is non-zero is a curve that surrounds a rotor.

% This is somewhat confusing

We can image that this curve follows the direction the wavefront
and by the time the wavefront get back at the starting position,
the state of each cell on the curve has been cycled through all states.

Another interesting insight can be found when rewriting Equation~\ref{eq:phase-index}.
Assuming that $\nabla\phi$ is smooth in a region $S$ that is bounded by a curve $C$,
Stokes' theorem can be applied, giving

\begin{equation}
  I = \frac{1}{2\pi} \iint_S \nabla \times (\nabla \phi) \cdot d\bm{S}
  \label{eq:phase-index-stokes}
\end{equation}

\noindent This expression will always be zero,
and thus, the only time that the phase index is different from zero will be when
$\nabla \phi$ is not smooth or if $\phi$ is undefined in some part of $S$ \autocite{herlin2012reconstruction}.
This is exactly the case when a rotor is present in $S$,
since at the rotor tip, the state is undefined.
We also like to mention that Equation~\ref{eq:phase-index-stokes}
is used in the kernel implementation of phasemapping found in (CITATION).

At this point, it should be clear how the phase index can be used to detect rotors,
but we would like to highlight some subtleties in its interpretation.
First, the phase index is a property of a closed curve
and not of a singular point.
Only when there is one singular point inside that curve,
a phase index can be assigned to that point.
Nevertheless, it is perfectly fine to draw a curve around multiple rotor tips
and calculate the phase index.
The phase index of this curve $I_C$ would then be equal to the sum of all rotor tips inside that curve.

\begin{equation}
  I_C = \sum_{\bm{x}_i\in C} I(\bm{x}_i)
  \label{eq:index-curve}
\end{equation}

Next, it should be mentioned that a non-zero phase index does not necessarily represent rotation of the wavefront.
While it is true that if there is rotational activity,
we can always find a curve with a non-zero phase index,
the phase index only indicates that travelling across that curve,
the phase has shifted with a multiple of the period.
To make this statement more concrete,
we have included a pattern in the simulation that is surrounded by a curve with a non-zero index,
but does not show any rotation.

Finally, we want to emphasize that the phase index is only defined if $\phi$ is smooth across $C$.
This brings us to the main topic of this paper.

\subsection{Phase Defects, Scar Tissue and Boundaries}

The requirement that $\phi$ is smooth across any curve is not trivial.
An obvious example is a curve that crosses a rotor tip,
but other patterns that do not involve rotor tips can also be found.
Like in \textcite{arno2021a, tomii2021spatial},
which discuss the existence of phase defects,
otherwise described as discontinuities in the phase field.
These discontinuities arise when the wavefront hits a region that is still in refractory,
forming a line where one side is excited and the other still side repolarizing.
These phases defects form our first problem,
since it is not possible to find a curve
that goes only around one point laying on the defect.
The best we can do is to draw a curve around this defect and calculated its phase index.

Another example where the $\phi$ is not smooth,
can be found in clinical practice.
When a patient has an arrhythmia,
the physician could perform ablation therapy
and burn  certain regions of the tissue.
The cells in these patches of scar have an undefined phase.
While the bulk of these patches don't show interesting behaviour,
the boundary still can.
However, just like with phase defect,
the best we can do is draw a curve around the non-conductive tissue
and calculate its index.

Finally, we should consider the boundaries.
This can be the outer boundaries of a 2D simulation,
or, a more clinical example, the mitral valve and pulmonary veins in the left atrium.
One can see the similarity between a non-conductive patch
and a hole in the tissue.
Therefore, boundaries can be treated similar as non-conductive patches.
This also enables us to capture patterns such as anatomical reentries~\autocite{duytschaever2024atrial}.
An overview on how to adjust the curves when phase defects and boundaries are present
is given in Figure~\ref{fig:index_calculation}
Notice, that the curve around the outer boundary is oriented in clockwise direction.
This convention has been proposed in \textcite{davidsen2004topological} to make sure
that the total phase index is conserved for compact surfaces.

\subsection{Conservation of Total Phase Index}

Let us first start of by looking at a closed surface $S$, meaning no boundaries,
and a phase field that is smooth except in a finite number of singular points $\bm{x}_i$.
If we now draw a curve on that surface with nothing in it,
its phase index will be zero as implied by Equation~\ref{eq:phase-index-stokes}.
Since \(S\) is a closed surface, this curve also encircles everything,
and since Equation~\ref{eq:index-curve} still applies,
we can write

\begin{equation}
  \sum_{x_i \in S} I(x_i) = 0
  \label{eq:index-theorem}
\end{equation}

\noindent This is the same as stating that the sum of all phase indices is
conserved on a closed surface.
This conservation law can be extended to compact surfaces \autocite{herlin2012reconstruction, davidsen2004topological},
resulting in the following equation:

\begin{equation}
  \sum_{x_i \in S} I(x_i) + \sum_{H_j \in S} I(H_j) + \sum_{D_k \in S} I(D_k)  = 0
  \label{eq:index-theorem-extended}
\end{equation}

\noindent where \(x_i\) are the singular points in \(S\),
\(H_j\) the holes,
and \(D_k\) the phase defects.
From now on we will refer to Equation~\ref{eq:index-theorem-extended} as the index theorem.

\begin{figure}[ht]
  \centering
  \includefig{index_calculation}
  \caption{Example of a bounded planar domain with holes. The domain
    contains singular points at \(\pmb{x}_1\), \ldots, \(\pmb{x}_n\) and
    holes \(H_1\), \ldots{} \(H_{m-1}\), with the outer boundary
    representing the final hole \(H_m\).
    Non-conductive patches are shaded in and considered as a hole.
    Curves are drawn around each singular point and hole
    with the arrow represent the direction of integration.
    In addition, a phase defect $D_k$ is presented
    together with a curve over which to integrate to get the phase index of the defect.
  }
  \label{fig:index_calculation}
\end{figure}

\vspace{2em}

\noindent Like stated above, the main topic of this paper is about the importance
of handling obstacles and boundaries in cardiac tissue.
The index theorem can be useful for assessing the quality of an analysis,
but it does not give a conclusive answer.
Therefore, in this paper we will try to present a simulation
that contains all pitfalls which are discussed above,
while still being simple enough to visually confirm the correct analysis.

\section{Methods}\label{methods}

There exists a numerous amount methods to detect rotational activity \autocite{pikunov2023the, gurevich2019robust, li2020standardizing},
making it is near impossible to create a comparative study or full review.
Therefore, we opted to give a demonstration of the effect
that boundaries and phase defect can have on the analysis.
This effect becomes most visible by comparing two approaches.
Both approaches will use a straightforward implementation of phasemapping, but
they will differ in the assumption that where made about the cardiac tissue.
In the first approach, it is naively assumed
that the phase is smooth across the complete tissue.
In the second approach, we account for possible obstacles and boundaries,
and perform a phasemapping analysis afterwards.

\subsection{Setup of the Simulation}\label{setup-of-the-simulation}

As it is not required to have realistic geometries to induce phase
defects, We chose to create a simulation on a homogeneous 2D grid. This
also has the benefit to get a full overview of the simulation at once
and easily place snapshots in the paper.
The simulation is created using Finitewave
(\url{https://github.com/finitewave/Finitewave}), an open-source Python
package for a wide range of tasks in modelling cardiac electrophysiology
using finite-difference methods. The main argument for using Finitewave
is its clear and transparent implementation of the models which allows
us to verify its correctness. Furthermore, the intuitive interface and
lots of examples make it easy to create, evaluate, and adjust
simulations.

The cell model used for the simulation is the Fenton-Karma model. Since
the research question of this paper does not involve the effect of ionic
channels on the dynamics, it is more fitting to use a phenomenological
model which is less computationally heavy and has simpler cell dynamics.
We choose the Fenton-Karma model specifically since it was the easiest
to create phase defects without tweaking the model's parameters.

The 2D mesh is first pre-paced with 10 planar stimuli and an interval of
200 time steps between them. Next, a boundary is added inside the mesh
and a second square stimulus is applied to induce rotational activity.
To ensure that the rotor fits on the mesh, but without increasing
computational time, we lowered the conductivity of the mesh. Finally, we
added a phase defect using a third stimulus in a small region just
before it goes out of refractory.

\subsection{Implementation of
phasemapping}\label{implementation-of-phasemapping}

To highlight the influence of phase defects and boundaries, we will
compare two implementations of phasemapping: one where the simulation is
scanned for singularities without addressing phase defects and ignoring
boundaries, which we will refer to as naive phasemapping, and another
were we take into account phase defects and boundaries. This second
implementation is basically an extension of the first one, so we will
refer to it as extended phasemapping.

\subsection{Naive Phasemapping}\label{naive-phasemapping}

The first step was to convert the action potential to a phase field.
This was done using taking the angle Hilbert transform of the action
potential as suggested by \autocite{bray2002considerations}. Additionally,
we made sure that the peak of the action potential corresponds to
\(\pi\) (see Figure~\ref{fig:action_potential_and_phase}),
so that it becomes straightforward to compare the phase field with
the action potential.

\begin{figure}[ht]
  \begin{center}
    \includegraphics[width=\columnwidth]{figures/action_potential_and_phase.png}
  \end{center}
  \caption{
    Time series of the action potential of one cell (orange)
    and the calculated phase (blue).
  }\label{fig:action_potential_and_phase}
\end{figure}

Next, we will create a triangulated mesh and compute the phase index for
each triangle at each time step. In theory, the cells of the mesh could
be any polygon, but triangulated meshes are quite common and choosing
so, made the code simpler.

For any polygon, the phase index can be calculated by counting the
number of phase jumps. An algorithm for this will look like:

\begin{enumerate}
    \def\labelenumi{\arabic{enumi}.}
    \tightlist
  \item
    Compute the phase difference \(\Delta\Phi\) of all edges.
  \item
    Count the number of times \(\Delta\Phi\) is bigger than \(\pi\)
    (positive phase jump).
  \item
    Count the number of times \(\Delta\Phi\) is smaller than \(-\pi\)
    (negative phase jump).
  \item
    Calculate the index with \(I = P - N\) with \(P\) and \(N\) the number
    of positive and negative phase jumps respectively.
\end{enumerate}

\subsection{Extended Phasemapping}\label{extended-phasemapping}

To extend the naive approach, the first thing to do is localizing the
phase defects. Remember that phase defects are defined as
discontinuities in the phase field. In a triangulated mesh this would
manifest as a big phase difference on the edges. Therefore, the most
straightforward thing to do is setting a threshold \(d\) so that an edge
with a phase difference \(\Delta\Phi\) that satisfies
\(d<\Delta\Phi<2\pi-d\) would be considered as a phase defect. We have
found that a value of \(d=0.08\pi\) works best for this simulation, but
keep in mind that this threshold depends highly on the parameters of the
simulation.

Once all phase defects are located, the cells that contain at least one
phase defect are removed. This will create holes in the mesh, allowing
us to treat phase defect and boundaries as the same.

Finally, the boundaries and holes are extracted as polygons. The naive
phasemapping approach (see previous section) is then applied to these
polygons together with the remaining cells.

\section{Results}\label{results}

Looking back at Figure~\ref{fig:comparison_snapshot_overview}, it is
clear that the naive
and extended approach does not give the result. Notice that the naive
approach localizes rotational activity in the bottom right and left
while the extended approach localizes rotational activity in the bottom
right and around the non-conductive region at the top. Moreover, the
analysis of the naive approach does not satisfy the index theorem given
in equation~\ref{eq:index-theorem-extended}. This should raise some
suspicion that this analysis
inaccurate.

\begin{figure}[ht]
  \centering
  \includegraphics[width=\columnwidth]{zoom_rotor.png}
  \caption{Zoom of the bottom left of the simulation, showcasing a
    counterclockwise rotation. Left: A snapshot at time step 172, with the
    colours represent the phase of the points at that time step. Rotors and
    critical cycles are annotated in red. Right: The phase density map taken
    across the entire time of the simulation. The colours represent the
    number of time steps that a point was annotated. A log scale was used to
    enhance visibility. The results of the naive approach are displayed at
  the top, and the results of the extended approach the bottom.}
  \label{fig:zoom-rotor}
\end{figure}

First, let's focus on the bottom left (see Figure~\ref{fig:zoom-rotor}).
Both approaches identify rotational activity in this region, and by
comparing both point density maps, it is clear that they also recognize
a similar drift. However, the extended approach detects a phase defect
at the centre of the rotational activity, indicating that this is a
linear rotor, while this information is not present in the analysis of
the naive approach.

\begin{figure}[ht]
  \centering
  \includegraphics[width=\columnwidth]{zoom_defect.png}
  \caption{Zoom of the bottom right of the simulation, showcasing a phase
    defect without rotation. From left to right, snapshots are taken at time
    steps 157, 177 and 235. The colours represent the phase of the points at
    that time step. (Counter)clockwise rotation is annotated in red (blue)
    and phase defect without rotation is annotated in white. The results of
    the naive approach are displayed at the top, and the results of the
  extended approach the bottom.}
  \label{fig:zoom-defect}
\end{figure}

Next, we shift focus to the bottom right (see
Figure~\ref{fig:zoom-defect}). Looking at the snapshots of the simulation,
the naive approach highlight two rotors in opposite direction, which
collide with each other before making a complete turn. In contrast, the
extended approach does not identify rotors in this region. Instead, it
finds a phase defect that is located between the two rotors. What
happened here is that the extended approach calculated the phase index
of both rotors together, and since these are of opposite sign, they
cancel each other out.

\begin{figure}[ht]
  \centering
  \includegraphics[width=\columnwidth]{zoom_reentry.png}
  \caption{Zoom of the top of the simulation, showcasing rotation around a
    non-conductive region. Only the results of the extended approach are
    shown, since the naive approach did not detect anything. From left to
    right and top to bottom, snapshots are taken at time steps 132, 172 and
    198. The colours represent the phase of the points at that time step and
  critical cycles with clockwise rotation are annotated in blue.}
  \label{fig:zoom-reentry}
\end{figure}

Finally, we look at the top of the simulation (see
Figure~\ref{fig:zoom-reentry}). We now have the opposite situation as
before at the bottom right: The extended approach finds rotational
activity around the non-conductive tissue while the naive approach does
not. A closer observation of the different snapshots reveals an
interesting pattern: The number of wavefronts around the non-conductive
tissue alternates between one and three. But what makes is even more
peculiar is that none of the wavefronts makes a complete turn. This
reminds us about the pattern seen in the bottom right where two rotors
collide with each other before completing a turn. Yet, this time the
total phase index is non-zero.

\section{Discussion}\label{discussion}

The presented simulation is nothing out of the ordinary.
It has only three main components:
a linear core, rotation around a non-conductive region and a phase defect.
That is it. No anisotropic conductivity, no fibres and no noise.
The simulation is also of a good resolution, which also favours phasemapping.
Yet, the result of the naive approach is quite different
from that of the extended approach.
We argue that the result of the extended approach is more favourable
as it aligns better with our visual inspection.

Additionally, the result of the naive approach does not fulfil the index theorem (Eq.~\ref{eq:index-theorem-extended}),
implying that the analysis is incorrect.
Indeed, the naive approach detects two rotors around the phase defect at the bottom right,
but there is no rotation present at that location.
Some people would dismiss these rotors tips
by stating that they do not make a full rotation,
but this argument would also dismiss the near-complete rotation
found around the scar tissue.
However, recent clinical studies indicate
that this pattern plays a critical role in atrial tachycardia~\autocite{duytschaever2024atrial, santucci2024identification, takigawa2019a}.
Its significance can be explained by the index theorem (Eq.~\ref{eq:index-theorem-extended}).
Because the sum of all indices is always zero,
one could hypothesize that ablation strategies
not affecting all structures with a non-zero index will not terminate the tachycardia.

Another interesting remark is
that the extended approach does not find any rotors.
This result aligns with the results from \autocite{tomii2021spatial},
that states that all rotors are found around a phase defect
This could imply that only finite cores are possible.
An implication that is also strengthened by the argument that the
tissue has a finite minimal wavelength,
or the smallest curve with a non-zero phase index has a finite circumference.
And since non-conductive structures, boundaries
and phase defects are all treated in the same manner,
it is only natural to call them all finite cores in case they have a non-zero index.

% Not sure if this is relevant
This absence of rotors in the extended approach's result also
indicates that the rotors found
by using the naive approach are all due to integrating across phase defects.
This could explain the surprising results in \textcite{li2020standardizing},
where it was found that the outcome of rotor detection
depends on the threshold for identifying phase jumps.
In case only real phase jumps where present,
the results would only vary when this threshold was set close to $2\pi$.
But phase defect could be affected by this threshold
at any value between $\pi$ and $2\pi$.

\vspace{2em}

\noindent To conclude this article,
we acknowledge that the described approaches is probably too simplistic
to analyse clinical or more complex data.
Nevertheless, it should be clear
that phase defects and non-conductive tissue have a major impact
on the analysis of cardiac arrhythmia, even in this simple example.
Therefore, we invite the reader to try to analyse this example with a state-of-the-art method
and see how many pre -and post-processing steps are necessary to get the correct solution.

In case an adventurous clinician has stumbled upon this rather technical paper,
we would like to refer them to our OSF page,
which contains a study protocol and necessary tools
to start a clinical study about localising finite cores during atrial tachycardia.

\section{Data availability}\label{data-availability}

\begin{itemize}
    \tightlist
  \item
    GitHub: code + source code
  \item
    Zenodo: figures, simulation data, video
  \item
    OSF: preregistration
\end{itemize}

Readers who would like to reproduce our results are referred to this
GitHub repository (LINK!!!), which contains the used code and some
further explanation.

\printbibliography
\end{document}
